\documentclass[a4paper,10pt,epsf,fleqn]{article}
%\documentclass[a4paper]{jsarticle}
\usepackage{graphicx}	% required for `\includegraphics' (yatex added)
\author{Takao Kotani}
\title{ecalj --- Get statrted (aug2013)} 
\usepackage{setspace}
\setstretch{1.1}
\begin{document}
\maketitle
\tableofcontents
\newpage
\section{Introduction}
``ecalj package'' is for first-principle electronic structure
calculations with unique features. 
Especially, as we explain in the next subsection,
a main feature is in the quasiparticle self-consistent GW (QSGW)
calculation based on the PMT method (=Linearized APW+MTO method).

\subsection{Features of the ecalj package.}
Generally speaking, central part in any electronic structure
packages is the one-body problem solver. 
That is, how to calculate eigenvalues and eigenfunctions for given one-body
potential. In turn, we can calculate one-body potential 
for given eigenfunctions and eigenvalues
based on the density functional theory (DFT) in the LDA or GGA
(hereafter LDA means both of LDA and GGA).
Then we can make the electron density self-consistent by iterations
until converged, and obtain total energy of ground state.
We can also calculate atomic forces simultaneously.
Based on such a solver, we can implement 
higher-level approximations such as the QSGW method, where we replace
the exchange-correlation potential in DFT with that given by a GW method.
In addition, kinds of physical properties, linear responses and so on, 
can be calculated.

Such one-body problem solvers (in the case of linear methods) are
characterized by (i) linear combinations of what basis set give eigenfunctions; 
(ii) how to represent electron density and one-body potential.
In the ecalj, we uses the PMT method, which is an
all-electron full potential method where we use not only the
augmented plane waves (APW), but also the muffin-tin orbitals (MTO),
in addition to the local orbital (lo). Within out knowledge no other
methods allows us to use two kinds of augmented waves simultaneously.
Thus eigenfunctions are represented by linear combinations of the
APWs, MTOs, and the lo's.
Then the electron density and the one-body potential are represented
by ``smooth part + onsite muffin-tin (MT) part - counter parts to
remove smooth part within MTs.
This part (Soler-Williams formalism) is essentially 
the same as that of the projected augmented wave method.

With ecalj, we can perform the GW calculation. 
The usual GW approximation is to obtain
quasiparticle energies (QPE) by one-shot calculation
starting from LDA; this is so-called ``one-shot GW''.
Its ability is limited; it can fail when its starting point 
(eigenfunctions and eigenvalues supplied by LDA) is problematic.
Thus T.Kotani with collaborators have developed the QSGW method. 
The QSGW now becomes popular, performed by other researchers. 
In principle, results by QSGW do not depend on LDA anymore; 
the LDA are only used to prepare initial condition for
self-consistency cycle of the QSGW calculation (however, 
exactly speaking, we use LDA to assist efficient implementation of QSGW,
e.g. to prepare required radial functions). 

Usually the QPEs obtained by QSGW reproduce experiments better than LDA.
For example, the band gap by GGA for GaAs is about 0.5 eV in contrast
to the experimental value of 1.6eV (If we virtually remove electron-phonon
effect from the experimental value, it becomes about 1.7eV). 
On the other hand, the QSGW predict about 1.9eV, slightly larger
than experiment (for practical use, we sometimes add 
``scaling correction'' on it to have better agreements with experiments).
Even in the case of NiO and so on, QSGW gives not so bad results (it
tends to give a little larger band gaps than experiments).

From QPEs and eigenfunctions obtained in QSGW, we can
calculate dielectric functions and so on.
But total energy in QSGW is still in research.

The ecalj package can also do other functions, 
LDA+U, atomic force and relaxation (in GGA/LDA), 
spin fluctuation, optical response and so on.

Recent development of 
dual-based MTO prescription allows us to use very localized MTOs
(with damping factors $\exp(-r/(1{\rm a.u.}))$), together 
with APWs of low energy cutoff ($\sim 3$ Ry). 
I think this is promising not only for efficient DFT/QSGW scheme, 
but also for kinds of applications in future.
The MTOs can be used instead of the Wannier functions,
but not so much research on it yet.

The QSGW calculation requires so much computational time:
roughly speaking, it takes $10$ or more times expensive than usual
one-shot GW (but you can reduce computational time by choosing
computational conditions).
Thus the size of systems we can treat is about ten atom in a cell;
computation requires about a week or so to have reasonable convergence.
(heavy atoms require longer computational efforts,
light atoms faster; we still have room to accelerate the method, but not
yet so much. Minimum MPI parallelization is implemented).
The computational effort is $\propto N^4$.

The ecalj web site is at 
\begin{verbatim}
    https://github.com/tkotani/ecalj
\end{verbatim}
We can download ecalj package from this site.
Free to download, modify, and use it, but we need you to clarify 
acknowledgment to the package in your publications; 
without it, we can not continue further developments.
We also have a web site at http://pmt.sakura.ne.jp/, 
however, most of all are yet in Japanese and not yet well organized.
The ecalj is related to lmv7 package at\\
\verb+http://titus.phy.qub.ac.uk/packages/LMTO/fp.html+. 
The lmv7 and ecalj are branched off at year 2009.
After branched, major contributions are due to Takao Kotani and Hiori
Kino (NIMS). All codes are now in f90, and we have simplified 
complications in the lmf7 code. 

A manual on lmv7 may be a help to learn ecalj. 
But it may make you confused, because we have removed confusing
parts from it, and simplified. Read this first. Then I show where we
should read in the lmv7 manual.

If you have something, let takaokotani@gmail.com know it.
With your ideas, we like to have collaborations and newer development on
it. Or I may help something on it.


\subsection{What is in this booklet?}
Here we can show minimum on the ecalj package.
\begin{itemize}
\item
How to perform self-consistent calculations by the density functional
theory (DF) in the LDA 
%\footnote{In the followings, DF usually means ``DF in LDA or GGA''}
\item
How to perform the QSGW calculations.
\item
How to plot energy bands (BAND), total density of states
(DOS), and the partial density of states (PDOS).
\item
How to plot dielectric functions, and non-interacting spin
     susceptibility $\chi_0^{+-}$. (not documented yet...)
\end{itemize}

To perform calculations becomes easier recently.
After we prepare a crystal structure file named as \verb+ctrls.*+,
we run a a script (\verb+ctrlgenM1.py+) to generate \verb+ctrl.*+
Its contains (not so bad) default setting to do calculations.
To help writing \verb+ctrls.*+,  ecalj contains samples 
and a converter between VASP-POSCAR format and ecalj-ctrls format.
(Thus we can use tools for VASP).

However, we may need to know a Little more to obtain results for publications;
some knowledge to judge whether obtained results 
are reasonable or not. And check convergence by some different conditions.
Especially, results by QSGW need to be examined carefully. 
But, an advantage is that the QSGW code is version controlled and hosted
by github, your results can be reproduced by others.


%%%%%%%%%%%%%%%%%%%%%%%%%%%%%%%%%%%%%%%%%%%%%%%%%%%%%%%%%%%%%%%%%%%%%%%%
\section{Install}
\label{install}
Look into ecalj/README. (or you can see it at https://github.com/tkotani/ecalj).
The installation is not do difficult (especially for gfortran, and ifort). 
After finished, we have all required binaries and shell scripts in your
~/bin. (or somewhere else where you specified at the bottom of Makefile.)
%After installed, run install checker to confirm correct
%installation.


%%%%%%%%%%%%%%%%%%%%%%%%%%%%%%%%%%%%%%%%%%%%%%%%%%%%%%%%%%%%%%%%%%%%%%%%%%%
\section{LDA calculations}
Calculations are performed by following steps.
\begin{itemize}
\item Write ctrls file,``crystal structure file'', which contains crystal structure.
  It can be by hand, or convert it from POSCAR (in vasp). There
  is a tool to convert between POSCAR and ctrls. (ecalj/StructureTool/README).
  There is a checker, lmchk, to confirm the crystal structure is read well or not.
\item Generate ctrl from ctrls by a script \verb+ctrlgenM1.py+. 
     \verb+ctrl+ is the main control file which contains all required
      information for calculations; as a part it contains ctrls.
      Then we we edit the generated ctrl file for your purpose.
\item Do lmfa (just calculate atoms (MT sites) placed in the cell). 
      It also calculate core eigenfunctions and valence charge for initial
      condition. And start main calculation lmf successively. 
      Then we get self-consistent result of LDA.
\item Plot energy band, DOS, PDOS, by running scripts.
\end{itemize}

These \verb+ctrls+ and \verb+ctrl+ are with extensions as
\begin{verbatim}
    ctrl.(id)
    ctrls.(id)
\end{verbatim}
. \verb+(id)+ is extension which we can add
(only lower letter allowed). For example, \verb+ctrls.cu+, \verb+ctrls.lagao3+. 



\subsection{Write crystal structure file, ctrls}
\label{ctrls}
%Let us explain how to write ctrls.
%Let us make a test directory, and move to it.
% %It might be better to move to a directory for your test, 
% e.g, move to \verb+~/ecaljtest+. (note that \~/\  means your root(home) directory).
% \begin{verbatim}
%     $ mkdir ~/ecaljtest
%     $ cd ~/ecaljtest
% \end{verbatim}
% Let us start from "fcc Cu". So, make a directory under \verb+~/ecaljtest+
% \begin{verbatim}
%     $ mkdir Cu
%     $ cd Cu
% \end{verbatim}

We have a sample of Cu in \verb+~/ecalj/lm7K/TESTsamples/Cu/ctrls.cu+; it is
\begin{verbatim}
% const da=0 alat=6.798
STRUC  ALAT={alat} DALAT={da}
       PLAT=  0.0 0.5 0.5  0.5 0.0 0.5   0.5 0.5 0.0
SITE    ATOM=Cu POS=0 0 0
\end{verbatim}

Another sample at \verb+~/ecalj/lm7K/TESTsamples/GaAs/ctrl.gaas+ is
\begin{verbatim}
#id  = GaAs
%const bohr=0.529177 a=5.65325/bohr 
STRUC
     ALAT={a} 
     PLAT=0 0.5 0.5  0.5 0 0.5  0.5 0.5 0 
SITE
     ATOM=Ga POS=0.0 0.0 0.0
     ATOM=As POS=0.25 0.25 0.25
\end{verbatim}

Lines starting from '\#' are neglected as comment lines.
Lines starting from \verb+% const+ define variables and set values
(in these cases, \verb+da+ and \verb+alat+ \verb+bohr+). 
Then the variable are used as \verb+{alat}+; in the cu case,
\verb+{alat}+ means 6.798.
Lines not start from "\#" nor "\%" are main content in the ctrls file.

Note that we have two tags of ``categories'' "STRUC" and "SITE". 
These tags should start from the first column. 
Thus ctrls is divided by multiple ``categories''.
In a category, we have ``tokens'' such as ALAT, DLAT, PLAT. 
These under STRUC category. 
ALAT+DALAT specify a unit to mesure length in this ctrl file.
These are in a.u. (= bohr radius=0.529177\AA). 
We use ALAT+DALAT as unit to specify primitive vectors to specify unit cell.
The unit cell is given by PLAT (ALAT as unit).
In the above example, three primitive cell vectors specified by nine
numbers after PLAT=; they give three primitive vectors;
PLAT1=(0,0,0.5), PLAT2=(0.5, 0.0, 0.5), and PLAT3=(0.5, 0.5, 0). 
DALAT is convenient to change lattice constant; but it is fixed to be
zero here; thus no effect in this example.

Note that SITE category can have multiple ATOM tokens. The number of
ATOM token under SITE should be  the same as atoms in the primitive cell.
In the case of GaAs; SITE contain multiple ATOM tokens.
POS just next to ATOM is taken as subtokens under ATOM token. 
(this may looks slightly uncomfortable since the end of range of ATOM
 token is not so clear; but it is not a problem).
In cases, we specify such subtokens as SITE\_ATOM\_POS.

In the SITE category, we place atoms (MT names) in the primitive cell.
In these cases we use defaults atomic symbol (MT names) for \verb+ATOM+.
POS is in the Cartesian coordinate, but in the unit of ALAT+DALAT.

For your test, you may make a test directory and copy a ctrls.* to your directory.
If you have VESTA and ecalj/StructureTool installed, you can see its structure by 
\begin{verbatim}
  $ viewvesta ctrls.cu
\end{verbatim}
(here \$ means command prompt).
\begin{quote}
NOTE: As written in ecalj/README, you have to install VESTA and viewvesta. 
Then set VESTA= at the top of ecalj/Structure/viewvesta, and make softlink to it.
The command \verb+viewvesta+(\verb+~/ecalj/StructureTool/viewvesta.py+)
generate \verb+POSCAR_cu.vasp+ first, then send it to VESTA.
\verb+viewvesta+ also accept \verb+POSCAR_cu.vasp+ directly.
Except names starting from \verb+ctrl+ and \verb+ctrls+,
\verb+viewvesta+ sends the name to VESTA directly. 
VESTA requires extension '.vasp' to identify VASP format.
We have samples in \verb+~/ecalj/StructureTool/sample+.
A tool \verb+vasp2ctrl+ converts POSCAR\_\*.vasp to ctrls.\*.
``--help'' show a small help. \\
$\bullet$ ecalj/StructureTool/ is not tested well. Not believe it so
much... We will fix it on your request.
\end{quote}

Another example with atoms is 
\verb+~/ecalj/lm7K/TESTsamples/SrTiO3/ctrls.srtio3+. This
contains \verb+SITE+ category as
\begin{verbatim}
    SITE
      ATOM=Sr POS=1/2 1/2 1/2
      ATOM=Ti POS= 0   0   0
      ATOM=O  POS=1/2  0   0
      ATOM=O  POS= 0  1/2  0
      ATOM=O  POS= 0   0  1/2
\end{verbatim}
. Note that an expression 1/2 can be used for POS. 
As it show, we can use mathematical expression instead of values.
Mathematical expressions such as ``$+ -  * /$ sqrt(...)'' are recognized.
(instead of $3**2$, use $3^2$. You can use parenthesis, but no space for separation).
(After ctrl generated, these is a command lmchk to check weather atomic
structure is correctly given or not.)
We can use default atomic symbols (to check default atom name (MT name) type \verb+ctrlgen2.py --showatomlist+).
Instead of such default symbols, we can use your own symbol as
\begin{verbatim}
    SITE
      ATOM=M1 POS=1/2 1/2 1/2
      ATOM=M2 POS= 0   0   0
      ATOM=O  POS=1/2  0   0
      ATOM=O  POS= 0  1/2  0
      ATOM=O  POS= 0   0  1/2
    SPEC
      ATOM=M1 Z=38
      ATOM=M2 Z=22
      ATOM=O  Z=8
\end{verbatim}
. Then we have to add extra category SPEC where we set Z number.
(You can use Z=37.5 for virtual crystal approximation, however, 
you can not do it in ctrls. Edit it in ctrl file.)\\

We can see other samples in \verb+~/ecalj/lm7K/TESTsamples/*/ctrls.*+.
(we also have a sample generator. See later section.)
Note that ctrls file is jut in order to generate default ctrl file in
the followings. Not from ctrls but from ctrl, we can start calculations.
Thus ctrls is not necessary if we prepare ctrl file directory.


\subsection{Generate default ctrl from ctrls by ctrlgenM1.py}
To run programs of lm7K (lmfa and lmf) in PMT,
we need an input file ctrl which contains many other settings.
To generate ctrl from ctrls, we have a command "ctrlgenM1.py" (written
in python 2.x and call fortran code internally).
Two steps required to complete ctrl file:
(i) we give reasonable options to invoke ctrlgenM1.py. 
This generate a ctrl file.;
(ii)we need to edit the ctrl file afterward for your calculation.

At first, try \verb+ctrlgenM1.py+ without arguments. It shows help. 
To generate \verb+ctrl+ from \verb+ctrl+, type
\begin{verbatim}
   $ ctrlgenM1.py cu --nk1=8
\end{verbatim}
Here cu specify ctrls.cu. The option --nk1=8 
means the number of division of the Brillowin zone for
integration. It means 8x8x8 division. If we like to use 8x8x4, 
we have to supply three arguments --nk1=8 --nk2=8 --nk3=4.
The above command gives following console output.
\begin{verbatim}
    $ ctrlgenM1.py cu --nk1=8
     === INPUT arguments (--help gives default values) === 
      --help  Not exist
      --showatomlist  Not exist
      --nspin=1
      --nk=8
      --xcfun=vwn   !(bh,vwn,pbe) 1
      --systype=bulk !(bulk,molecule)
      --insulator  Not exist !(do not set for --systype=molecule)

    ...

    OK! A template of ctrl file, ctrlgen2.ctrl.cu, is generated.
\end{verbatim}
As we see above, 
options which you specified are shown at the beginning of the console output
(in this case --nk1=8). Others such as --nspin=1 are default settings.
If we like to perform spin-polarized calculations, we add other option
'--nspin=2' as
\begin{verbatim}
    ctrlgenM1.py cu --nspin=2 --nk1=8
\end{verbatim}
(NOTE:In the spin-polarized case, we need to set initial condition of size of
magnetic moment at each atoms: we have to edit
SPEC\_ATOM\_MMOM of ctrl file (\verb+MMOM=s p d f ...+) to be like
\verb+MMOM=0 0 2+. We will explain this later on).
The \verb+ctrlgenM1.py+ generates ctrl file named as
\verb+ctrlgenM1.ctrl.cu+. To do calculations, copy it to ctrl.cu so 
that lmf recognize it.
\begin{verbatim}
   cp ctrlgenM1.ctrl.cu ctrl.cu
\end{verbatim}

\subsection{crystal structure checker: lmchk}
Do lmchk to confirm correct crystal structure is really given or not.
\begin{verbatim}
   lmchk --pr60 cu 
\end{verbatim}
Then it reads \verb+ctrl.cu+. \verb+--pr60+ is an option of verbose. Bigger number gives more information.
\begin{itemize}
\item Lattice info, Space group symmetry operations (in lmf format), and
      their generators (these operations can be generated from a few of them.)
      See \verb+http://titus.phy.qub.ac.uk/packages/LMTO/tokens.html#SYMGRPcat+
      about how to represent the operations.
\item Show atomic positions. If rst.* (this is generated after DFT
      calculation and relaxation) exists, it may show a position in
      rst.* file (relaxed position). NEED to CHECK.
\item Tabulate MT radius and distance between atomic sites.
\end{itemize}

\subsection{ctrl file}
The ctrl file generated by the \verb+ctrlgenM1.py+ contains explanations. 
Read it first. This is a head part of \verb+ctrl.cu+ generated by ctrlgenM1.py:
\begin{verbatim}
    ### This is generated by ctrlgenM1.py from ctrls 
    ### For tokens, See http://titus.phy.qub.ac.uk/packages/LMTO/tokens.html. 
    ### Do lmf --input to see all effective category and token ###
    ### It will be not so difficult to edit ctrlge.py for your purpose ###
    VERS    LM=7 FP=7        # version check. Fixed.
    IO      SHOW=T VERBOS=35 TIM=2,2
                 # SHOW=T shows readin data (and default setting at the begining of 
    console output)
                 # It is useful to check ctrl is read in correctly or not
                   (equivalent with --show option).
                 # larger VERBOSE gives more detailed console output.
    SYMGRP find  # 'find' evaluate space-group symmetry automatically.
                 # Usually 'find is OK', but lmf may use lower symmetry
    ...
\end{verbatim}
\verb+#+ means comment lines. We can also use
lines start from \verb+% const ...+ to define variables and set constant.

%The ctrl file has a grammatical structure (although not completely systematic).
We see ``categories'' such as \verb+VERS+ and \verb+IO+, and so on.
The beginning of categories are starting from the first column.
Under categories, we have "tokens" such as \verb+VERBOSE+.
Thus we specify full name of token \verb+VERBOSE+ under category
\verb+IO+ as \verb+IO_VERBOSE+.

The ctrl file generated by ctrlgenM1.py contains explanations.
Thus read them first. Here we give some complementary explanations to it.

\begin{itemize}
\item
\verb+IO_TIM+ is for debugging. It shows which subroutines are
called and so on. Bigger number shows deeper subroutines.

\item
\verb+SYMGRP+ is a category without token under it; 
we set generators of space group (See explanation in previous paragraph).
When we set \verb+find+, it automatically calculate symmetry of crystal lattice.
If we like to enforce symmetry, set some of generators which are shown by lmchk.

\item
We see \verb+ctrls+ is embedded in the \verb+ctrl+ by \verb+ctrlgenM1.py+.
\begin{verbatim}
    ... (skip) ...
    % const  da=0 alat=6.798
    STRUC   ALAT={alat} DALAT={da}
            PLAT=  0.0 0.5 0.5  0.5 0.0 0.5   0.5 0.5 0.0
            NL=4 NBAS= 1  NSPEC=1
    SITE    ATOM=Cu POS=0 0 0
    ... (skip) ...
\end{verbatim}
NL, NBAS(number of SITE) and NSPEC(number of SPEC) are automatically
added by ctrlgenM1.py.
It is possible to deform unit cell by adding optional tokens
(it is possible to rotate PLAT for magnetic anisotropy calculation).
See \verb+http://titus.phy.qub.ac.uk/packages/LMTO/tokens.html#STRUCcat+.
For new calculations, it is better to find some examples first.

\item
{\bf SITE} category:
As for MT sites, we have two categories.
(1)SPEC(species) and (2)SITE(specify centers of atoms(species) in primitive cell)
As for SPEC, we specify MTs(radius, Z, MTOs on it) appeared in the cell.
These are defined subtokens under SPEC\_ATOM=foobar (we have multiple SPEC\_ATOM=foobar).
Then we place these SPEC\_ATOM=foobar at SITE sections.

At SITE, we specify atomic sites 
(What SPEC\_ATOM is placed to positions by POS) in a primitive cell.
We set \verb+POS=+ by direct form (Cartesian) but with the unit of \verb$ALAT+DLAT$.
Total number of SITE (number of tokens SITE\_ATOM) is
the number of atoms in the primitive cell.
SITE\_ATOM=foobar means we place foobar defined in SPEC\_ATOM=foobar.
We set subtoken SITE\_ATOM\_POS under SITE\_ATOM, to specify atomic positions.
In addition, we can set SITE\_ATOM\_RELAX, if you like to find relaxed
structure (we simultaneously set DYN category) in LDA. As for
relaxation, see \verb+LaGaO_relax/ctrl.lagao+
example, and read\\ \verb+http://titus.phy.qub.ac.uk/packages/LMTO/tokens.html#DYNcat+.\\

The SITE\_ATOM=foobar (with same foobar with different POS) are not
necessarily equivalent with respect to the space group operation of a system.
%On the other hand, atoms belonging to a CLASS(this is not in ctrl file) 
%is equivalent on space group operations. 
Thus \verb+SITE_ATOM=foobar+ are divided into ``classes'' which are
connected by the operation. 
The lmf automatically judge ``classes'' (see also info by lmchk). 
Thus not need to specify it, but it may be better to check it.
A sample is \verb+lmchk lagao+ at \verb+~/ecalj/lm7K/TESTsamples/LaGaO_relax+

\item
{\bf SPEC} category: 
In ctrls, we have not yet specified contents of SPEC; 
we have just given default symbols or only Z= when we use non-default
names (shown by ctrlgenM1.py --showatomlist).
The command \verb+ctrlgenM1.py+ adds default SPEC sections.

We have some \verb+SPEC_ATOM+, 
under which we give subtokens such as
\verb+SPEC_ATOM_R+(MT radius), \verb+SPEC_ATOM_Z+(nucleus charge), 
cutoff parameters of angular momentum, and so on. 
These \verb+SPEC_ATOM+ is refereed to in SITE.

An example of SPEC category is
\begin{verbatim}
SPEC                                                            
    ATOM=Fe Z=26 R=1.70 
      KMXA={kmxa}  LMX=3 LMXA=4 NMCORE=1                        
      PZ=0,3.9,4.5
      EH=-1 -1 -1 -1  RSMH=0.85 0.85 0.85 0.85          
      EH2=-2 -2 -2   RSMH2=0.85 0.85 0.85
      MMOM=0 0 2 0                                                    
  
    ATOM=... (then the similar block of ATOM= are repeated.)
       ...
\end{verbatim}
Under the token \verb+ATOM=Fe+, we have subtokens
\verb+SPEC_ATOM_Z+,\verb+SPEC_ATOM_R+, and so on.

Subtokens Z= is the nucleus charge and R= MT radius.
Note that Fe is just a name to distinguish MT sphere in the cell.
If you set SPEC\_ATOM\_Z=27, it is recognized as Co (since Z=27). 
\verb+LMX=3+ is the maximum l of MTOs. Thus maximum l of MTO is l=3.
The maximum of l to expand electron density and potential within MT is
LMXA (in contrast to usual LAPW), we can use quite small LMXA such as
LMXA=4. NMCORE=1 means we calculate core density without non magnetically-polarization.
This can reduce computational confusion.

PZ is to set local orbital (if not, no local orbitals). EH and RSMH are
to specify first set of MTOs.(We can check how local orbitals are 
set by lmfa explained in the next section).
EH2 and RSMH2 are to specify second set of MTOs. 

After PZ=, we have three numbers.
These are numbers for s,p,d,f,g,... channels. Zero means not exist.
You can use space or , as delimiter. 
Here not only the integer part of principle quantum number, but also 
the fractional part should be supplied (If PZ=0,3,4, it does not work.)
Now PZ=3.9 for p and PZ=4.5 for d. This means we use local orbital for
3p, and local orbital for 4d (fractional parts (continuous principle quantum number) 
are large $\sim 0.9$ for core like orbital, and smaller for extended
orbital $\sim 0.3$ or something. See Logarithmic Derivative Parameters
at \verb+http://titus.phy.qub.ac.uk/packages/LMTO/lmto.html+).
This is a little confusing, thus we will explain this in appendix. See Sec.\ref{xxx}.

EH(damping factor), and RSMH (where the smooth Hankel function bent)
determines MTOs. We now set four numbers for them. Thus we set MTOs
s,p,d,f with EH=-1 and RSMH=0.85. Our current test shows that RSMH is
one half of R (that is, 0.85=1.70/2, but minimum RSMH is 0.5) 
and not need to be dependent on s,p,d,f. (If LMX=2, s,p,d are allowed and no f MTOs.)
EH is -1; not need to change except test purpose.
In a similar manner, EH2 and RSMH2 for second set of MTOs are given.
Just three numbers means these for s,p,d. 

MMOM=s,p,d,f... gives initial condition of magnetic moment in $\mu_B$
(number of up-down electron).

In cases such as As, the local orbital given by default ctrl
is responsible of rather deep core, and it is not need to be treated 
as valence electrons. In such a case,
we don't need local orbital.

In the case of AntiFerro-II NiO, it contains 
two NiO in a primitive cell. Thus it is reasonable to have place two SPEC\_ATOM
as Ni1 and Ni2, although subtokens under
ATOM=Ni1 and ATOM=Ni2 (e.g. \verb+SPEC_ATOM_EH+ for them) are the same
except initial condition of magnetic moment of MMOM=s,p,d,f...
See example of NiO.
\end{itemize}

%%%%%%%%%%%%%%%%%%%%%%%%%%%%%%%%%%%%%%%%%%%%%
The minimum help of call Category\_token\_subtoken are listed with
minimum explanation with 
\begin{verbatim}
    $ lmf --input
\end{verbatim}
It gives a long output. But many of them are experimental and not need
to touch them. A part of it is
\begin{verbatim}
 Token            Input   cast  (size,min)
 ------------------------------------------
    ... ...

     STRUC_ALAT        reqd   r8       1,  1
       Scaling of lattice vectors, in a.u.
    ... ..
\end{verbatim}
This is an minimum explanation of it. "reqd" means "required" (no
default). r8 means it read with real number, 1,1 means
that ALAT=xxx should contain one number minimum (max is also one)
(See also STRUC\_PLAT, and so on).

There are kinds of examples in ecalj packages.
Please look into their of ctrls.* and ctrl.*
These are in lm7K/TESTsample/* and ecalj/CMDsampls. 
In addition, ecalj/MATERIALS contain many samples; see a later subsection. 

In anyway, we think it is very important to add examples which shows all
the functions of ecalj package (not only materials but also to show its
functions). It is not yet, but we will do in on your request.
%Type \verb+lmf --input+ (no side effect) shows minimum explanation.
%It lists all Category\_Token\_Subtoken; but it maybe too much for beginners.

As for what is shown in \verb+$ lmf --input+, most of important tokens are
already described in the ctrl file generated by ctrlgenM1.py.
So, we don't need to care many options shown by it.
But we have not explained 
``STRUC category to deform crystal, DYN category for dynamics, LDA+U
treatment, Adding background charge, Core-Hole treatments'' here. 
We will prepare examples for it if requested.
\begin{verbatim}
http://titus.phy.qub.ac.uk/packages/LMTO/tokens.html#STRUCcat
http://titus.phy.qub.ac.uk/packages/LMTO/tokens.html#DYNcat
\end{verbatim}

\subsection{Do LDA/GGA calculations, and get convergence}
\label{lm7K-scf}
Here we show how to get converged results from a \verb+ctrl+ file.

At first, we need initial guess of charge density.
It can be given by a super position of atomic charge density.
To obtain the charge density, we solve atoms first. It is by
\begin{verbatim}
   $ lmfa gaas | tee llmfa
\end{verbatim}
It takes just a few seconds. Here tee is a command of Linux.
It divide console output (standard output) to a file, and to console.

Then try
\begin{verbatim}
    $ grep conf llmfa
\end{verbatim}
. Then you see a key point that
\begin{verbatim}
conf:SPEC_ATOM= Ga : --- Table for atomic configuration ---
conf:  isp  l  int(P) int(P)z    Qval     Qcore   CoreConf
conf:    1  0       4  0         2.000    6.000 => 1,2,3,
conf:    1  1       4  0         1.000   12.000 => 2,3,
conf:    1  2       4  3        10.000    0.000 => 
conf:    1  3       4  0         0.000    0.000 => 
conf:    1  4       5  0         0.000    0.000 => 
conf:-----------------------------------------------------
conf:SPEC_ATOM= As : --- Table for atomic configuration ---
conf:  isp  l  int(P) int(P)z    Qval     Qcore   CoreConf
conf:    1  0       4  0         2.000    6.000 => 1,2,3,
conf:    1  1       4  0         3.000   12.000 => 2,3,
conf:    1  2       4  3        10.000    0.000 => 
conf:    1  3       4  0         0.000    0.000 => 
conf:    1  4       5  0         0.000    0.000 => 
conf:-----------------------------------------------------
\end{verbatim}
This is an initial electron distribution, and how we divide 
core and valence. In this case core charge Qcore are (6 electron for s channel=1s,2s,3s
and 12 electron for 2s and 3p). Core is not treated separately from valence electrons
(frozen core approximation; we superpose rigid core density to make
all-electron density). Qval means electrons for each s,p,d channels.
The valence channels are 4s,4p,4d,4f (when we set EH=s,p,d,f). 
The int(P)z column is for local orbital. Thus we have 3d treated as
local orbital. (our code can add one local orbital per l)

isp means spin (1 or 2), since --nspin=1 for Ga and As, no isp=2 exist.
In summary we have 4s,4p,4d,3d,4f as valence. 
This means we use corresponding number of MTOs and local orbitals.

After lmfa, let us start main calculation.
\begin{verbatim}
    $ lmf cu
\end{verbatim}
It might be better to do
\verb+$ lmf cu | tee llmf+
in order to keep console output to llmf. 
As it is a iteration calculations, 
it shows similar output again and again.
Then you end up with self-consistent result as
\begin{verbatim}
    ......
   it  8  of 30    ehf=   -3304.895853   ehk=   -3304.895853
 From last iter    ehf=   -3304.895856   ehk=   -3304.895855
 diffe(q)=  0.000003 (0.000007)    tol= 0.000010 (0.000010)   more=F
c ehf=-3304.8958531 ehk=-3304.8958529
 Exit 0 LMF 
 CPU time:    7.024s     Mon Aug 19 02:03:19 2013   on  
\end{verbatim}
\verb+it  8  of 30+ means it stop at 8th iteration, although we set
maximum number of iteration 30. Note that this number is 
given by \verb+ITER_NIT=30+ in \verb+ctrl.cu+).
\verb+ehf+ and \verb+ehk+ are the ground state energy in Ry.
They are calculated in a Little different manner. Although
they are different during iterations, it finally get to be the
almost the same number.(but it can be different like 10 micro Ry per atom
even converged. But you don't need to care it so much).
ehk:Hohenberg-Kohn energy, ehf: Harris-Faulkner energy.

``grep diffe lllmf'' shows how the changed of total energy (and charges)
during iteration. diffe(q) mean  changes of energy with previous
iteration、q is for electron density difference as well.
See also save.* file, which only show ehk and ehk.

Thus we do have ground state energy.
Although output of lmf is long, most of all are to monitor
convergence.
As long as it converged well, you don't need to look into it in detail.
Eigenvalues are shown as
\begin{verbatim}
 bndfp:  kpt 1 of 4, k=  0.00000  0.00000  0.00000   ndimh = 122
 -1.2755 -1.2008 -1.2008 -0.2052 -0.2052 -0.2052 -0.0766 -0.0766 -0.0766
 -0.0174 -0.0174 -0.0174  0.1094  0.1095  0.1095  0.2864  0.2864  0.4170
  0.4170  0.4736  0.6445  0.6445  0.6445
\end{verbatim}
This is at k=  0.00000  0.00000  0.00000 .
(because of historical reason, two same bndfp: are shown in each
iteration; two band path method). At,  ``lmf cu|grep -A6 BZWTS'' shows the Fermi energy
(for insulator, we see band gap). 
Deep levels which gives little dependence on k are core like levels.

rst.* contains is the main output which contains electron density.
mix.* is a mixing file (which keeps iteration history).
When you restart lmf again, it read rst.cu and mix.cu.
If you start from lmfa result, please remove them.
(or lmf --help show option of --rs=(five numbers), 
how to read atm file which is the initial atom file by lmfa).
We can do parallel calculation with lmf-MPIK, 
we can invoke it with mpirun -np 8 lmf-MPIK cu. It should give the
same answer.


%%%%%%%%%%%%%%%%%%%%%%%%%%%%%%%%%%%%%%%%%%%%%%%%%%%%%%%%%%%%
\subsection{DOS, Band, PDOS plot}

We already have script to plot dos and band from the result of lmf
self-consistent calculations.
At \verb+~/ecalj/lm7K/TESTsmaples/*+, we have
\begin{verbatim}
    job_tdos, job_band_nspin1, job_band_nspin2
\end{verbatim}
. Read this script, and then you see how to plot energy bands.

For total DOS plot, it is better to check ctrl file;
\verb+BZ_TETRA=1+(this is default; thus make sure that \verb+BZ_TETRA+ do
not exist or \verb+BZ_TETRA=1+). 
In addition, we have to enlarge number of k point \verb+NKABC+ large enough.
Then we do
\begin{verbatim}
    job_tdos cu
\end{verbatim}
This shows total DOS as
\begin{figure}[h]
 \begin{center}
  \includegraphics[width=70mm]{img/ps.dos.cu}
  \vspace{5mm}
  \caption{DOS(Cu)}
 \end{center}
\end{figure}
 

For band plot, we have to set symmetry lines along which we plot eigenvalues.
It is given in \verb+syml+. \verb+syml+ also must have extension
as \verb+syml.cu+. Collections are in ecalj/MATERIALS/. Choose and
modify one of them and rename it.

Here let us a sample in syml.cu.
\begin{verbatim}
    $ cp ~/ecalj/lm7K/TESTsamples/Cu/syml.cu .
\end{verbatim}
Thus we have \verb+syml.cu+ to your directory.

Look into \verb+syml.cu+; it is
\begin{verbatim}
    21  .5 .5 .5     0  0 0           L to Gamma
    21   0  0  0     1  0 0          Gamma to X
    0    0 0 0  0 0 0
\end{verbatim}
First line means, we calculate eigenvalues 
for {\bf k} points from {\bf k}=(0.5,0.5,0.5) to {\bf k}=(0,0,0).
"L to Gamma" is just a comment since program only read 
seven numbers for each line.
Second line means, we calculate eigenvalues 
for k points from {\bf k}=(0,0,0) to {\bf k}=(1,0,0).
3rd line means calculation just stop here.
Units of {\bf k} are in 2$\pi$\verb+ALAT+.

A line starting from '\#' is neglected (comment line).

Do
\begin{verbatim}
    job_band_nspin1 cu
\end{verbatim}
is for spin=1. \verb+job_band_nspin2+ is for spin=2 when it exist.
(These scripts try to determine the Fermi energy first. You may skip it in cases.)

\begin{figure}[h]
 \begin{center}
  \includegraphics[width=70mm]{img/ps.band.cu}
  \caption{band plot(Cu)}
 \end{center}
\end{figure}

For PDOS plot,
\begin{verbatim}
    job_pdos cu
\end{verbatim}
It shows many figures in gnuplot. It is a little expensive because we
use no symmetry to distinguish all lm channel in a simple manner.
(PDOS is not yet implemented for SO=1 case; spin-orbit coupling $\L\dot S$ is added.)
We have to re-organize script of gnuplot (pdos.site*.*.glt) for your purpose.
In principle, meanings of all data files are shown (see at the bottom
of console output about lm ordering in a line), thus not so difficult to
rewrite *.glt. For example, to plot eg and t2g separately. 
(NOTE: site id is shown by lmchk).\\

\noindent {\bf \large WARNING}: Usually lmf and so on recognize options such as -vnspin=2 or
something, where nspin is defined as \verb+% const nspin=1+ in ctrl file.
The option overlaid \verb+% const nspin=1+; replace it with =2;
(this is shown as save.* file, and top of console output).
However, job\_tdos and so on, do not yet accept these options. Thus 
you may need to modify these command (we will fix it in future.)


%\paragraph{Practice}
%Let start from scratch. 
%Make directory as \verb+~/ecaljtest/Si+ 
%And make \verb+ctrls.si+ first.
%Then generates \verb+ctrl.si+, and run lmfa, lmf successively. 
%There are other sample. Try it by yourself.


\subsection{Samples: ecalj/MATERIAL/}
We have a material database in \verb+ecalj/MATERIALS/+.
At the directory, type  
\begin{verbatim}
  $ ./job_materials.py
\end{verbatim}
Then it shows a help. You see 
\begin{verbatim}
...
=== Materials in Materials.ctrls.database ===
  2hSiC  3cSiC  4hSiC  AlAs  AlN  AlNzb  AlP  AlSb  Bi2Te3  C
  CdO  CdS  CdSe  CdTe  Ce  Cu  Fe  GaAs  GaAs_so  GaN
  GaNzb  GaP  GaSb  Ge  HfO2  HgO  HgS  HgSe  HgTe  InAs
  InN  InNzb  InP  InSb  LaGaO3  Li  MgO  MgS  MgSe  MgTe
  Ni  NiO  PbS  PbTe  Si  SiO2c  Sn  SrTiO3  SrVO3  YMn2
...
\end{verbatim}
. For these simple materials, input files will be generated.
The ctrls are stored in \verb+ecalj/MATERIALS/Materials.ctrls.database+
(in addition, options passed to ctrlgenM1.py and options to lmf-MPIK are included).
The command \verb+./job_materials.py+ gives ctrls.* for these materials
based on the description in the \verb+Materials.ctrls.database+.
And then it generates ctrl file by calling ctrlgenM1.py internally, 
and run lmfa lmf-MPIK successively (when no --noexec).

For example, try
\verb+./job_materials.py Fe --noexec+. (not fe but Fe as it shown above).
Then it makes a directory Fe/ and set ctrl.fe (also ctrls.fe) in the
directory. Without '--noexec', it does calculation for Fe successively.
As for NiO and Fe, we see that \verb+./job_materials.py+ gives
SPEC\_ATOM\_MMOM in generated ctrl file.

Try \verb+job_materials.py GaAs Si+.\\
Then directories GaAs/ and Si/ are generated. See \verb+save.*+ files containing
total energies iteration by iteration. Starting from \verb+ctrl.*+ in
these directory, the command perform DFT calculations 
(Console output is stored in \verb+llmf+, \verb+save.*+ gives
total energies. \verb+rst.*+ contains self-consistent
density, from which we can calculate energy bands and so on).

``\verb+./job_materials --all --noexec+'' generates ctrls and ctrl files of
these materials. ``\verb+./job_materials --all+'' do self-consistent
LDA calculations for materials (it takes an hour or more. Change the
number of cores for MPIK in the script(search lmfjob) if you like).\\


To make band plot and so on for Fe,
\begin{verbatim}
  $ ./job_materials.py Fe  (and need to type return)
  (If you like start over, remove Fe/ under it first).
  $ cd Fe
  $ ./job_materials.py fe
    (but it might be better to do --noexec, and observe Fe/ctrls.fe and
	Fe/ctrl.fe first. grep conf llmf shows the initial electron distribution).
  $ cat save.fe  (this shows total energies of each iteration. 'c ' at
	the first column gives converged result. 'h ' is from atm file.)
        If it does not ends with 'c ...' line, something strange
	occurs. see llmf (console out put of lmf is saved to llmf).
  $ cp ../syml.fe .
  $ job_band_nspin2 fe 
        (As I said, this shell script do not yet accept
	    options to lmf. Look into the script).
        (This calculate fermi energy first for safe; it takes
	    some time)       
  $ job_tdos fe
  $ job_pdos fe (as I said, this supress space-group symmetry, thus time consuming).
\end{verbatim}
Then it shows a help. 
See joblmf file also (it contains options to invoke lmf. This is shown
in save.*. In principle, options in joblmf should be passed to band plot
and so on. But not yet implemented (it is not so difficult).



%%%%%%%%%%%%%%%%%%%%%%%%%%%%%%%%%%%%%%%%%%%%%%%%%%%%%%%
\newpage
\section{QSGW calculation}
The QSGW calculation requires to calculate ``non-local exchange-correlation
potential $\Sigma_{\rm xc}$'' (it may be better to say static version of self-energy), which is
calculated by GW calculation. Then $\Sigma_{\rm xc}-V_{\rm xc}^{\rm
LDA}$ is stored into \verb+sigm+ file. Then, we calculate one-body
calculation by lmf (or lmf-MPIK) where we add sigm to one-body
potential. This iteration cycle is performed by a script ``gwsc'' as we
explain later on.

\subsection{GWinput}
Let us start from ctrls.si as
\begin{verbatim}

\end{verbatim}

\label{GWinput}
In order to perform QSGW, one another input file 
\verb+GWinput+ is necessary in addition to \verb+ctrl+.
Thus all input files for QSGW is just two files, ctrl.* and GWinput.
A template \verb+GWinput+ can be generated by a script \verb+mkGWIN_lmf2+. 
You may have to modify it in cases for your purpose.

Let us start from ctrls.si;
\begin{verbatim}
#id  = Si
%const bohr=0.529177 a= 5.43095/bohr
STRUC
     ALAT={a} 
     PLAT=0 0.5 0.5  0.5 0 0.5  0.5 0.5 0 
SITE
     ATOM=Si POS=0.0 0.0 0.0
     ATOM=Si POS=0.25 0.25 0.25
\end{verbatim}
. Do ``ctrlgenM1.py si --tratio=1.0''and rename ctrlgenM1.ctrl.si to
ctrl.si. (--tratio=1.0 means touching MT (check it by lmchk); 
it is probably slightly advantageous in our version of GW calculation).
Let us invoke \verb+mkGWIN_lmf2+ as follows.
\begin{verbatim}
    $ lmfa si (lmfa is needed to do in advance).
    $ mkGWIN_lmf2 si
    ......
    == Type three integers n1 n2 n3 for Brillowin Zone meshing for GW! ==
     n1=
\end{verbatim}
Then it pause and ask numbers. You have to type three numbers as
2+ return + 2+return+2 return.
\begin{verbatim}
    == Type three integers n1 n2 n3 for Brillowin Zone meshing for GW! ==
     n1= 2
     n2= 2
     n3= 2
    2 2 2
    ...(skip)...
    OK! GWinput.tmp is generated!
\end{verbatim}
Generated file is \verb+GWinput.tmp+; you have to copy it to \verb+GWinput+.
\begin{verbatim}
    $ cp GWinput.tmp GWinput
\end{verbatim}
These '2 2 2' you typed is reflected in a section 'n1n2n3 2 2 2 ' in \verb+GWinput+.
You can edit it, and change it to e.g. 'n1n2n3 4 4 4' if you like to
calculate self-energy on dense BZ mesh. (QPNT.chk contains irreducible k
point for given n1 n2 n3; KPTKPTin1BZ.gwinit.chk contain all k points in
Brillowin Zone). Generally speaking, you don't need to repeat mkGWIN\_lmf2 
as long as you don't change MTO sections in ctrl file (number of EH,EH2,PZ).

Look into \verb+GWinput+. Because of historical reason,
input system is different from ctrl.
Each line is independent; ``keyword'' followed by some of numbers 
(or on or off for logical switch of keyword). 
Except such lines, there are tag sandwiched sections 
such as \verb+<PRODUCT\_BASIS> ... </PRODUCT_BASIS>+, 
no comment lines in the middle allowed. They are read by
read(*,*) (thus spaces cause no problem).

In the QSGW calculation, we have to set some cutoff parameters
for self-energy calculations.

\begin{itemize}
\item
\verb+emax_sigm+ is the maximum limit of the self-energy 
(measured from the Fermi energy). I think $2.5\sim5$Ry is reasonable choice.
But in cases, small \verb+emax_sigm+ can give poor dispersion curve
(slightly unnatural behavior) because of sudden cutoff by emax\_sigm.
However, we like to use smaller value to reduce computational time.
That is, larger is better, but expensive. 
Generally speaking, accuracy less than 0.1eV (for bandgap) 
is allowance of current method. Probably, it may be not impossible to have better accuracy, 
but it may ask us to repeat many calculations with changing conditions
to confirm stability. 

\item
\verb+0.100000D-02 ! =tolopt+ controls a number of Product basis
to expand the Coulomb interaction. (The product basis is to expand the
Coulomb interaction is different from the basis to expand eigenfunction.)
In our experience, \verb+0.100000D-02+ (=0.001) is not so bad.
If you like to reduce computational time use 0.01 or so, but a little
dangerous in cases. With 0.0001, we can check stability on it.

\item
     \verb+QGcut_psi+ is a little (usually 0.5 or so) larger than \verb+QpGcut_cou+.
     It becomes accurate if we use large \verb+QpGcut_cou+. 
     But it enlarge size of IPW(interstitial plane wave) part of Mixed
     product basis. For test, try  2.7, 3.2, 3.7 for \verb+QGcut_cou+
     (and add 0.5 or 1 for \verb+QGcut_psi+). Larger one is expensive.
\item
     \verb+dw+ and \verb+omg_c+ specify real space bins which we accumulate imaginary part
     weight of polarization functions. dw is bin width (in Ry) at
     omega=0, then bin width is twiced at \verb+omg_c+.
     The bin width is quadratically larger (become rough). 
     If bins are too wide, dielectric function can be less accurate, 
     but results are not necessarily so much affected. 
     For metal, our code can capture Drude weight
     numerically. We do not need to be so sensitive to the choice of
     them usually.
\item
  n1n2n3: BZ division for k point integration. 
     We usually take '4 4 4' to '8 8 8' for GaAs. For metal such
     as Fe, '10 10 10' or more is better.
\item
     lcutmx(atom) is the l cutoff of product basis for atoms 
     in the primitive cell (do lmchk for atom id).
     In the case of Oxgen, we can usually use lcutmx=2 (need check by
     the diffence when you use lcutmx=2 or lcutmx=4). 
     Then the computational time is reduced well.
\item
     Other part of product basis section in \verb+<PRODUCT\_BASIS>... </PRODUCT\_BASIS>+
     is usually not need to be touched. But if you
     like to calculate big systems with smaller CPU time,
     or do very accurate calculations, we may need to touch it.  
     It is described elsewhere.
\item
    \verb+<QPNT>+ tag is to specify one-shot GW. At which point, do we
    calculate QPE, not for QSGW. If you set k point in it not on regular mesh point,
    you have to set 'Any Q on'; but it is expensive.
    Since QSGW have ability to plot energy band within full BZ,
    it should be better to do it.
\item
   \verb+<QforEPS>+ and \verb+<QforEPSL>+ are to specify at which k point do we
   calculate susceptibility. It is for epsPP\_lmfh, eps\_lmfh
   (dielectric functions) and epsPP\_chipm (spin susceptibility).
\end{itemize}

\begin{quote}
We need a setting in ctrl file to read sigm file (HAM\_SIGP). 
It is simplified now, and not need to care it so much.
As we set RDSIG=12 in defaults, lmf read sigm file and add it to one-body
potential as long as sigm.* exist.\\

{\bf NOTE for old users}: We now set \verb+SIGP[MODE=3 EMAX=9999.]+
in ctrl file to read self-energy in lmf (or lmf-MPIK). 
This is because we use very localized MTOs (similar with the Maxloc Wannier).
Our test shows reasonable results and this simplify algorithms.
In my previous version, we asked you to use \verb+SIGP[MODE=3 EMAX=2.0]+
where EMAX is a little (0.5Ry) less than \verb+emax_sigm+. If something
strange occurs, try this setting).
\end{quote}

In principle, QSGW result should not depended on the choice of XCFUN.
However, it can affect slightly. In our tests, it seems slightly better
to use vwn (XCFUN=1) for QSGW calculations. (BUT need to check more...)


%%%%%%%%%%%%%%%%%%%%%%%%%%%%%%%%%%%%%%%%%%%%%%%%%%%%%%%%%%%%%%%%%%%
\subsection{do QSGW calculation}
\label{fpgw-calc}
Let us perform QSGW calculation. 
For this purpose, we use a script \verb+gwsc+. 
\begin{verbatim}
    gwsc (number of iteration+1) -np (number of nodes) (id of ctrl)
\end{verbatim}
If (number of iteration+1)=0, it gives one-shot calculation from LDA.
But it is different from the usual one-shot in literatures;
since it calculates off-diagonal elements of self-energy also,
we can plot energy band dispersion plot. In cases (such as usual
semiconductors), it gives rather reasonable results in comparison with
experiments from practical point of view.

This is an example of one iteration of QSGW cycle.
(now a little different but essentially similar)
\begin{verbatim}
takao@TT4:~/ecalj/test1$ gwsc 0 -np 2 si
gwsc 0 -np 2 si
### START gwsc: ITER= 0, MPI size=  2, TARGET= si
--- No sigm nor sigm.$TARGET files for starting ---
 ---- goto sc calculation with given sigma-vxc --- ix=,0
No sigm ---> LDA caculation for eigenfunctions 
        Start  mpirun -np 2 /home/takao/ecalj/TestInstall/bin/lmf-MPIK  si > llmf_lda 
OK! --> Start  echo 0| /home/takao/ecalj/TestInstall/bin/lmfgw si > llmfgw00 
OK! --> Start  echo 1|/home/takao/ecalj/TestInstall/bin/qg4gw > lqg4gw 
OK! --> Start  echo 1|mpirun -np 2 /home/takao/ecalj/TestInstall/bin/lmfgw-MPIK  si> llmfgw01 
OK! --> Start  /home/takao/ecalj/TestInstall/bin/lmf2gw >llmf2gw
OK! --> Start  echo 0|/home/takao/ecalj/TestInstall/bin/rdata4gw_v2  > lrdata4gw_v2 
OK! --> Start  echo 1| /home/takao/ecalj/TestInstall/bin/heftet > leftet 
OK! --> Start  echo 1| /home/takao/ecalj/TestInstall/bin/hchknw > lchknw 
OK! --> Start  echo 3| /home/takao/ecalj/TestInstall/bin/hbasfp0 > lbasC 
OK! --> Start  echo 3| mpirun -np 2 /home/takao/ecalj/TestInstall/bin/hvccfp0 > lvccC 
OK! --> Start  echo 3| mpirun -np 2 /home/takao/ecalj/TestInstall/bin/hsfp0_sc > lsxC 
OK! --> Start  echo 0|/home/takao/ecalj/TestInstall/bin/hbasfp0  > lbas 
OK! --> Start  echo 0| mpirun -np 2 /home/takao/ecalj/TestInstall/bin/hvccfp0  > lvcc 
OK! --> Start  echo 1|  mpirun -np 2 /home/takao/ecalj/TestInstall/bin/hsfp0_sc > lsx 
OK! --> Start  echo 11|  mpirun -np 2 /home/takao/ecalj/TestInstall/bin/hx0fp0_sc > lx0 
OK! --> Start  echo 2|  mpirun -np 2 /home/takao/ecalj/TestInstall/bin/hsfp0_sc  > lsc 
OK! --> Start  echo 0|  /home/takao/ecalj/TestInstall/bin/hqpe_sc  > lqpe 
OK! --> == 0 iteration over ==
OK! --> Start  mpirun -np 2 /home/takao/ecalj/TestInstall/bin/lmf-MPIK  si > llmf_gwscend.0 
OK! ==== All calclation finished for gwsc 0 -np 2 si ====
\end{verbatim}
Here \verb+echo (integer)+ is readin in at the begging of the code.
To see it, please look into gwsc script (gwsc is at
ecalj/fpgw/exec/ and copied to your bin/ by make install2). 
In anyway, this console output shows calculations finished normally.\\

Now we get rst.si and sigm.si file which contains (static version of) self-energy
minims $V_{\rm xc}^{\rm LDA}$.
What we did is the one-shot GW from LDA result; but note that we
calculate not only diagonal elements but also off-diagonal elements. 

We can write energy dispersion (band plot) in the same manner in LDA.
To do it, we need rst.si, sigm.si, ctrl.si, QGpsi, and ESEAVR.
(but QGpsi and ESEAVR are quickly reproduced). After you have syml.si
(e.g. in ecalj/MATERIALS/), Do
\begin{verbatim}
    $ job_band_nspin1 si
\end{verbatim}

\begin{figure}[h]
 \begin{center}
  \includegraphics[width=70mm]{img/ps.band.oneshot.si}
  \caption{Si, one-shot GW with off-diagonal elements}
 \end{center}
\label{sigwscone}
\end{figure}
You can observe large band gap as shown in the Fig.\ref{sigwscone}.
(To see it again, \verb+gnuplot bnds.gnu.si -p+.
All plots are in gnuplot, thus it is easy to replot it as you like).

We have QPU file (and also QPD for spin=2), which contains content
of the diagonal part of self-energy. It will be explained elsewhere.

You can make total DOS and PDOS plot by 
\begin{verbatim}
    $ job_tdos si
    $ job_pdos si
\end{verbatim}
CAUTION:pdos plot is not allowed for so=1. (even tdos--> ask to t.kotani.)\\

To get final QSGW results, we have to repeat iteration 
until eigenvalues are converged.
Note that total energy shown by console output llmf (and also shown in
save file) is not so meaningful in the QSGW; we just take it as an indicator to check convergence. 
Let us repeat 5 iteration more. "-np 2" means one core to use.
\begin{verbatim}
$ gwsc 5 -np 2 si
### START gwsc: ITER= 5, MPI size=  2, TARGET= si
 --- sigm is used. sigm.$TARGET is softlink to it  ---
 ---- goto sc calculation with given sigma-vxc --- ix=,0
 we have sigm already, skip iter=0
 ---- goto sc calculation with given sigma-vxc --- ix=,1

 ...(skeip here) ...

         mpirun -np 2 /home/takao/bin/hsfp0_sc  < _IN_ > lsc 
OK
         /home/takao/bin/hqpe_sc  < _IN_ > lqpe 
OK
5 iteration over
   mpirun -np 2 /home/takao/bin/lmf-MPIK  si > llmf_gwscend.5
\end{verbatim}
Note that we do skip 0th iteration (it is for one-shot from LDA) since
we start from rst.si and sigm.si given by one-shot LDA.
Thus we do just five iterations.
Information of eigenvalues are in \verb+QPU.(number)run+ files.
(for magnetic systems with nspin=2), wee have \verb+QPD.(number)run+ also).
Check it by ls;
\begin{verbatim}
    $ ls QPU.*run
    QPU0.run  QPU.1run  QPU.2run  QPU.3run  QPU.4run  QPU.5run
\end{verbatim}
(These are overwritten when we again repeat gwsc; be careful.)
Note that \verb+QPU0.run+ was old one when you did 1-shot GW from LDA 
at the beginning.

In order to check convergence calculations going well, do
\begin{verbatim}
   $ grep gap llmf*
\end{verbatim}
This shows how band gap changes in llmf.*run files.

Let us check convergence of the QSGW calculations.
For this purpose, it is convenient to take a difference of QPU(QPD) files
by a script \verb+dqpu+. These files are human readable.
To compare \verb+QPU4.run+ and \verb+QPU5.run+, do
\begin{verbatim}
    $ dqpu QPU.3run QPU.4run
\end{verbatim}
Then we see a list of numbers (these are the differences of values in
QPU files).  Then it shows at the bottom as
\begin{verbatim}
    Error! Difference>2e-2 between:   QPU.4run   and   QPU.5run  
    :  sum(abs(QPU-QPD))= 0.05736
\end{verbatim}
but you don't need to care it so much.
You rather need to check the difference of values.
I can say most of all difference (especially around the Fermi energy are
) are almost 0.00eV or 0.01eV, we can judge QPEs are converged.
If not converged well, you may need to repeat 
\verb+gwsc+ again.
(when the size of two QPU files are different, dqpu stops.)

\begin{figure}[h]
 \begin{center}
  \includegraphics[width=60mm]{img/ps.band.oneshot.si}
  \includegraphics[width=60mm]{img/ps.band.gga.si}
  \caption{band plot(Si, QSGW one-shot test)    and    band(Si) (GGA)}
 \end{center}
\end{figure}

%\paragraph{practice}
%Perform QSGW calculation for GaAs.
%Try one-shot GW, and plot it first.
%Then do QSGW. You can try 'n1n2n3 4 4 4', it is a little time
%consuming (maybe about 30 minutes per iteration or more).

%\begin{figure}[hbtp]
%  \includegraphics[width=60mm]{img/ps.dos.gga.si}\hspace{10mm}
%\end{figure}

\begin{figure}[hbtp]
  \includegraphics[width=60mm]{img/ps.band.qsgw.gaas}
  \caption{band(GaAs), QSGW (test case)}
\end{figure}

\section{calculation of dielectric function}
  eps mode.


%%%%%%%%%%%%%%%%%%%%%%%%%%%%%%%%%%%%%%%%%%%%%%%%%%%%%%%%%
\section{Appendix}
\subsection{appendix 1; How to add spin-orbit coupling}
\begin{verbatim}
Do LDA and/or QSGW with SO=0 first.
Then apply the spin-orbit coupling by perturbation.

After converged with nspin=1 (or 2), create new directory and copy
  ctrl.gas, rst.gas, sigm.gas, QGpsi, ESEAVR
to it. Then we set
  nspin=2 
  METAL=3
  SO=1  (this is ldots calculation off-diagonal elements included).
  Q=band (we do not change potential.)
in ctrl.gas. 
Then run
>lmf gas >& llmf_SO
You can see "band gap with SO" by 
> grep gap llmf_SO.
Then you can see two same lines.
 VBmax = 0.101949  CBmin = 0.236351  gap = 0.134402 Ry = 1.82786 eV
 VBmax = 0.101949  CBmin = 0.236351  gap = 0.134402 Ry = 1.82786 eV
(two lines are because of two-band path mechanism, 
which asks less memory usage than a path mehod)
This is the band gap with SO as a first-order perturbation starting on
top of the "QSGW without SO". When you use ctrl file generated by
ctrlgenM1.py. You can do the above procedure with
>lmf --rs=1,0 gas -vnit=1 -vso=1 -vnspin=2 -vmetal=3 --quit=band)
(--rs=1,0 read rst.gas but not write rst.gas. Ses lmf --help
 -vso=1 replace the setting of % const so=0 with so=1).

For band plot, you can use the same procedure 
for the case without SO. (Look into the job_band_nspin1 script.
You have to modify it so that --rs=1,0 gas -vnit=1 -vso=1 -vnspin=2
-vmetal=3 --quit=band is added as arguments for >lmf --band:syml ...).

---
For given sigm file, it is possible to do full self-consisent SO calculations
(then we do not set Q=band). However, note that Vxc is fixed in QSGW, 
it is not necessary better than the above procedure.

\end{verbatim}

\subsection{appendix 2: How to set local orbitals}
\begin{verbatim}
  As we stated, do "lmfa |grep conf" to check used MTO basis. 

  We have to set SPEC_ATOM_PZ=?,?,? 
  (they ordered as PZ=s,p,d,f,... ) to set local orbitals.
  
  lmv7 (originally due to ASA in Stuttgart) uses a special terminology
  "continious principle quantum number for each l", which is just
  relatated to the logalismic derivative of radial funcitons at MT
  boundary. It is defined as
   P= principleQuantumNumber + 0.5-1/pi*atan(r* 1/phi dphi/dr),
  where phi is the radial function for each l.   For example, 
   P= n.5 for l=0 of free electron (flat potential) because phi=r^0,
   P= n.25 for l=1 because phi=r^1; 
   P= n.147584 for l=2 because phi=r^2; P=, n.102416 n.077979 for l=3,4.
  (Integer part can be changed). See Logarithmic Derivative Parameters in
  http://titus.phy.qub.ac.uk/packages/LMTO/lmto.html#section2

  Its fractioanl part 0.5-atan(1/phi dphi/dr) is closer to unity for
  core like orbital, but closer to zero for extended orbitals.

  Examples of choice:
  Ga p: in this case, choice 0 or choice 2 is recommended.
      We usually use lo for semi-core, or virtually unoccupied level.

     (0)no lo (4p as valence is default treatment without lo.)
        3p core, 4p valence, no lo: default.
        Then we have choice that lo is set to be for 3p,4p,5p.
     (1)3p lo ---> 4p val (when 3p is treated as valence)
        3d semi core, 4d valence  
        Set PZ=0,3.9 
        (P is not requied to set. *.9 for core like state. It is just an initial condition.)
     (2)5p lo ---> 4p val (PZ>P)
        Set PZ=0,5.5 
        5.5 is just simply given by a guess (no method have yet
		implemented for 
        If 5.2 or something, it may fails
        because of poorness in linear-dependency. We may need to observe
        results should not change so much on the value of PZ.

     (3xxx)4p lo ---> 5p val (we don't use this usually. this is for test purpose)
        4p lo, 5p valence 
        Set PZ=0,4.5 P=0,5.5 (In this case, set P= simultaneously).
        (NOTE: zero for s channel is to use defalut numbers for s)

  Ga d: (in this case, choice 0 or choice 1 is recommended).
     (0)no lo (3d core, 4d valecne, no lo: default.)
          Then we have choice that lo is set to be for 3d,4d,5d.
     (1) 3d lo ---> 4d val  (when 3d is treated as valence)
         Set PZ=0,0,3.9  (P is not required to set)
     (2) 5d lo ---> 4d val  (PZ>P)
         Set PZ=0,0,5.5
     (3xxx) 4d lo ---> 5d val  (this is for test purpose)
         Set PZ=0,0,5.5 P=0,0,4.5
         (NOTE: zero  for s,p are to use defalut numbers )

   If you like to read from atm.ga file instead of rst file(if exist).
   You have to do lmf --rs=1,1,0,0,1, for example. See lmf --help
   Becase rst file keeps the setting of MTO, thus change in ctrl is not
   reflected without the above option to lmf.
=============================================================
\end{verbatim}


%%%%%%%%%%%%%%%%%%%%%%%%%%%%%%%%%%%%%%%%%%%%%%%%%%%%%%%%%%%%%%%%
\section{ MEMO random}
These are memo randoms. 
I have to explain them.

xxx under construction xxxxxxxxxxxxxxx

\begin{verbatim}
== not meaningful total energy ===
 Total energy shown in QSGW mode in current version is not meaningul.
 (just treat as an indicator to convergence).

== Do we use vwn or gga for QSGW? ===
  In prinicple, QSGW results should not depends on vwn or gga 
  (XCFUN=1 or 103 in ctrl). But there is minor dependence, because
   1. frozen core density.
   2. core eigenfunction.
   3. radial basis funcitons
   4. Slight numerical reason 
      (This is probably because Sigma-interpolation procedure
       But not exactly figured out yet
       -->affect about 0.02eV as for band gap for GaAs. ).
  In anyway, use vwn (HAM_XCFUN=1) as standard.
  And such technical things affects, 0.05 eV level of error for band gap.

== one show QSGW ==
  one-shot QSGW may be useful in cases.
  As it contains off-diagonal part, we can resolve band tanglement
  problem in Ge (no band gap).
 

== Restart calculation in lda ==
 lmf(lmf-MPIK) read rst.* in defaluts.
 rst contains electron density.
 If rst is already converged, it stops after two iteration.
 rst contains atomic positions.
 So, in order to read atomic positions change in ctrl,
 Use options shown in lmf --help.

== Restart calculation in qsgw ==
  To remove mixsigm* (mixing for sigm), maybe required.

== iteration check ===
  First, watch console output of gwsc (do redirect to output file)
  Need to check OK! signs arrayed on 1st columns.

  gwsc iteration is time cosuming,
  So we need to check calculations are normally going on or not.

  Memory inefficiency.
  Set 'KeepEigen off' an 'KeepPPOVL' off.
  In fact, out code is still inefficient for memory usage.
  
  grep gap llmf ---> minimum gap at mesh point.
  see save.* ,or grep '[xc] ' save.*
  the end of iteration of lmf is shown as x or c.
  (if failed, QPU file. 
  dqpu QPU.4run QPU.3run
  As for usual semi-conductor, accuracy abou t0.1 eV is limit of current implementation.
  Set vwn (xcfun=1) looks better (stable) for GW.

  $grep rms lqpe* 
  shows
           ...  rmsdel=2.44D-04
           ...  rmsdel=4.91D-03
           ...   rmsdel=2.44D-04
           ...   rmsdel=3.37D-04
  If rsmdel is getteing to be smaller, it is on convergence path.
  (but in magnetic cases, it may give be too good even not yet going to
	be converged..., beccause magnetic energy is so small)

  grep diffe llmf  ---> difference of energies of each iteration.

  ehf (harris energy)
  ehk (Hohenberg kohn energy)

== emax cutoff for APWs. ==
  We can not use so many APWs in current version,
  because of overcompleteness (this is because null vector within MTs), 
  In anyway, use pwemax=3 as standard (test it with 4 or 5).
  To avoid failure of calculation, we may use smaller MT radius for
  alkali, and alkali-earth elements. 

== Check Used MTO 
 Near beginig of console output, what MTO you use is shown as: (GaAs case).
 sugcut:  make orbital-dependent reciprocal vector cutoffs for tol= 1.00E-06
 spec      l    rsm    eh     gmax    last term    cutoff
  Ga       0*   1.13  -1.00   6.579    1.19E-06    1459
  Ga       1*   1.13  -1.00   7.028    1.26E-06    1807
  Ga       2*   1.13  -1.00   7.475    1.09E-06    2109
  Ga       3    1.13  -1.00   7.920    1.06E-06    2637
  Ga       0*   1.13  -2.00   6.579    1.19E-06    1459
  Ga       1*   1.13  -2.00   7.028    1.26E-06    1807
  Ga       2    1.13  -2.00   7.475    1.09E-06    2109
  As       0*   1.18  -1.00   6.300    2.13E-06    1243
  As       1*   1.18  -1.00   6.720    1.26E-06    1471
  As       2*   1.18  -1.00   7.140    1.37E-06    1837
  As       3    1.18  -1.00   7.558    1.05E-06    2229
  As       0*   1.18  -2.00   6.300    2.13E-06    1243
  As       1*   1.18  -2.00   6.720    1.26E-06    1471
  As       2    1.18  -2.00   7.140    1.37E-06    1837

== gwsc cause error stop.
 Have you ever changed MTO setting? Consistent with GWinput?

== QSGW for Fe.
  It is better to use 3p as core. Furthermore, 3d+4d as valence is better. 
  Thus we need to set PZ=0,3.9,4.5
  I also got aware that emax_sigm should be large enough (4$\sim$5 Ry) 
  to have smooth band dispersion. n1n2n3 can be 10x10x10.

== RSRNGE: enlarge RSRNGE ===
  Use RSRNGE=10 or so (in cases, RARNGE=20 or more is required), 
  for large number of k points. Try and enlarge it if it fails with a
  message "Exit -1 rdsigm: Bloch sum deviates more than allowed tolerance (tol=5e-6)".
  We will have to make it automatic in future.

== Q0P check
   In cases, it is better to use Q0Pchoice=2 instead of default Q0Pchoice=1.
   (For slabs, Q0Pchoice=2 may be better; need check more. In anyway,
    it is problematic to use unbalanced k points for anisotropic cell).

== EPS mode,
  Check Im part of chi0 is smoothly damping at high energy (typically
  1Ry or larger enengy range). If there is some large Im part remains,
  something strange (usually due to orthogonality problem of
  eigenfunctions when you set low q).

   Related source codes are in ecalj/lm7K/ .
   A command ecalj/lm7K/ctrlgenM1.py can generate 'standard input file (ctrl file)' 
   just from a given crystal structure file called as ctrls file. 
   Binaries are lmf and lmf-MPIK (MPI k-parelell verion).

   Recently, I renewed some part of algolism of GW/QSGW calculations
   (some ideas are taken from from PRB.81,125102(2010) 
    and Copmuter Physics Comm. 176(2007)1-13).
   ---> this is better than old versions; speed, memory (file size),
   and accuracy for anisortopic systems.
   For comparison, you can use old version in .git (gitk --all and check it out).

=== When calculation in LDA level fails ===
when calculation fails in LDA level.
  (1) smaller MT
  (2) fewer PW. smaller pwemax.
  (3) core as semicore.


=== LDA+U ===
not yet...
sample

=== MAE by rotating crystal ===
(we have a sample at

=== spin wave ===
J calculation.


xxxxxxxxxx
At line 663 of file slatsm/fopna.F (unit = 13, file = '')
Fortran runtime error: File 'vessm.gas' already exists
--> mpirun -np 12 lmf ? (you have to use lmf-MPIK..)

xxxxxxxxxxxxxxxxxxxxxxxxx
If not stable convergence in gwsc, try to set
mixbeta 0.5
(and/or mixpriorit 3 or something)
at the begining of sigma.


\end{verbatim}


\end{document}
